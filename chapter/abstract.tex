\chapter*{Abstract}
\label{chap:abstract}

This thesis deals with the development of an open source model generating feed-in time series from run-of-the-river hydropower plants. The objective being to run simulations based on open access geodatabases, the available input data was investigated. The resulting minimum input data set is the capacity and location of the plant, runoff time series over the period to simulate and historical runoffs over a few decades, as well as basic geographic data (boundaries, course of rivers...). All this data is available on the OpenEnergy Database, or will soon be. The chosen methodology for the model is to use the hydropower plants production equation and to extrapolate the unknown plant parameters (nominal head and water flow, type of turbine...) using the historical runoff time series. A first model was developed and tested on several study cases to asses its quality. Even though it does not produce satisfying results yet, the study cases allowed to identify sources of errors, and suggestions of improvement are advanced to pave the way to a more precise and reliable version of the model.


\chapter*{Zusammenfassung}

Diese Arbeit befasst sich mit der Entwicklung eines Open-Source-Modells zur Generierung von Einspeisezeitreihen aus Laufwasserkraftwerken. Da das Ziel ist, Simulationen basierend auf freien Geodatenbanken auszuführen, verfügbare Eingabedaten wurden untersucht. Der resultierende minimale Datensatz ist die Nennleistung und Standort der Kraftwerk, die Abflusszeitreihen der simulierten Periode so wie historische Abflusszeitreihen über ein paar Jahrzehnte, und elementare geographischen Daten (Grenzen, Flussläufe...). Dieser Datensatz ist verfügbar auf der Open Energy Database, oder wird bald sein. Die gewählte Methodik für das Modell ist, die Produktionsgleichung einer Laufwasserkraftwerke zu benutzen, und die unbekannten Kraftwerksparameter (Nennfallhöhe, Nennabfluss, Turbinenart...) von den historichen Abflusszeitreihen zu extrapolieren. Ein erstes Modell wurde entwickelt und getestet durch einige Fallstudien, um seine Güte zu bewerten. Obwohl die Ergebnisse noch unbefriedigend sind, ermöglichten die Fallstudien die Fehlerquellen zu ermitteln, und Verbesserungsempfehlungen werden vorangebracht, um eine genauer und zuverlässiger Fassung des Modells zu schaffen.


