\chapter*{Abstract}
\label{chap:abstract}

This thesis deals with the development of an open source model generating feed-in time series from run-of-the-river hydropower plants. The objective was to develop a generic model able to generate feed-in time series based on open-access data, even with very little information about the plants. The available input data was investigated and the relevant data sets for power plants, river data, and geographic data such as boundaries or course of rivers, are introduced. Part of this data was, where possible, made available on the OpenEnergy Database. The developed model is described in detail in this work. An approach using the hydropower plants production equation was chosen; unknown plant parameters such as nominal head and water flow or type of turbine being extrapolated using historical runoff time series. To assess the quality of the model three study cases where chosen, ranging from the simulation of a single power plant with known plant parameters to a state-wide simulation based on limited data. The validation showed important differences up to \unit[50]{\%} between real and simulated run-of-the-river energy production. Possible reasons for these differences are analysed, and approaches to improve the results are presented, among which better assignment methods of power plants to rivers and runoff time series and a finer definition of the exceedence percentage.


\chapter*{Zusammenfassung}

Diese Arbeit befasst sich mit der Entwicklung eines Open-Source-Modells zur Generierung von Einspeisezeitreihen von Laufwasserkraftwerken. Das Ziel war es, ein generisches Modell auf Basis frei verfügbarer Daten zu entwickeln, welches auch für geringe Kenntnisse der Kraftwerksparameter Einspeisezeitreihen berechnet. Es werden zunächst relevante Datensätze zu Kraftwerks- und Abflussdaten sowie geographischen Daten wie Grenzen und Flussläufen vorgestellt. Ein Teil dieser Datensätze wurde, soweit möglich, auf der Open Energy Database verfügbar gemacht. Im weiteren wird das entwickelte Modell detailliert beschrieben. Als Modellansatz wurde die Produktionsgleichung eines Laufwasserkraftwerks verwendet; unbekannte Kraftwerksparameter wie Nennfallhöhe, Nennabfluss und Turbinenart werden von historischen Abflusszeitreihen abgeleitet. Zur Validierung des Modells wurden drei Beispiele gewählt, die von der Betrachtung eines einzelnen Kraftwerks mit bekannten Kraftwerksparametern bis zu der Betrachtung eines Bundeslandes mit reduzierter Datenlage reichen. Die Validierung ergab, dass die vom Modell berechneten eingespeisten Energiemengen starke Abweichungen von bis zu \unit[50]{\%} der realen Einspeisung aufweisen. Mögliche Gründe für diese Abweichungen werden analysiert und Ansätze für potenzielle Verbesserungen, wie beispielsweise eine verbesserte Methodik für die Zuordnung eines Kraftwerks zu einem Fluss bzw. einer Abflusszeitreihe sowie eine verbesserte Abschätzung der Überschreitungshäufigkeit, vorgestellt.
