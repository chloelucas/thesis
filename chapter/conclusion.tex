\chapter{Conclusion}

The objective of this work was to develop with Python an open source model of run-of-the-river hydroelectric power plants able to simulate energy generation time series from open access geodatabases. In particular, the model was to be able to run simulations based on the Open Energy Database, to be consistent with the rest of the work carried out by the ``Transformation of Energy Systems'' team of the Reiner Lemoine Institute. \newline
The choice made in this work was to base the model on the typical equation of hydropower (Eq. \eqref{eq_power_1}) and to extrapolate where necessary missing data about the plants by integrating the steps of the design process for a run-of-the-river plant into the model (Sec. \ref{sub:assumptions}). Thereby, the nominal and residual water flows can be calculated from the flow duration curve on the site of the plant, and an optimal turbine type can be chosen from the nominal head and water flow. In this light, the model meets its basic requirements: it is freely available on github, along with use examples to facilitate the first steps of the user (Ch. \ref{chap:simulation_model}), it does not require more precise data on power plants than what is available in the OEDB, and it produces feed-in time series from hydropower plants. Additionally, the proper way to store runoff data in the OEDB has been discussed and implemented, along with the preprocessing of this data and of the plants registers, in order to assign each plant with consistent runoff time series (Sec. \ref{sec:data_preproc}). \newline
However, the simulations run with this model did not give satisfactory results yet (Ch. \ref{chap:results}). This is due in part to the lack of comprehensive and consistent data to compare with the results, as discussed in Sec. \ref{sec:db_hydroelec}, and in part to the simplifications made in the model and to the assignment of runoff time series to plants. The problems causing the uncertainties have been identified in Sec. \ref{sec:limits} and workarounds have been suggested in Sec. \ref{sec:improv}. \\ 

This project will continue within the openFRED project of the Reiner Lemoine Institute in order to improve the results of the model. The suggestions made in Sec. \ref{sec:improv} can be used to obtain a more precise output. Later on, when the open source modelled runoff data is made available, it will be uploaded into the OEDB and the use examples and preprocessing steps will be adapted to work with this data.\newline
One of the main problems while assessing the quality of the results was the very limited availability of production data from run-of-the-river power plants. It would be extremely interesting for the next steps of the project to have access to production time series of single plants with daily or hourly timesteps, as well as yearly state-wide production specifically from run-of-the-rivers plants. Concerning power plants registers, the publication of the Marktstammdatenregister in the months to come will hopefully provide more precise and comprehensive inputs for the model.
