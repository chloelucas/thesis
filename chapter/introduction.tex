\chapter{Introduction}
\label{chap:introduction}

\section{Energy systems modelling}
Pollution, global warming, depletion of fossil ressources, conflicts about access to fossil fuels --> energy transition \newline
Renewable energies (wind, solar, hydro, waves, geothermy, biomasse) --> decentralized, intermittent (for some of them) and highly dependant on Standort, weather, relief, etc.... \newline
Complexification of the energy system :\newline
* Supply doesn't follow demand (supply is not steuerbar)\newline
* No general solution applicable everywhere\newline
* Supply not produced where demand is\newline
--> use of modelisation amd simulations to find an optimal energy mix / distribution schema / optimal energy system\newline
Weather data for each standort\newline
Reliable and consistent info about Standort (existing plants, potential and space for new plants, flexibility)\newline
Energy conversion models\newline
Energy distribution models\newline
Energy demand data (load curve)\newline
Optimisation tools\newline

\section{Open source}
Publication of results, not tools --> same models developped by different actors --> not efficient  \newline
Growing volume of research about RE + tools available to share and work together + complexity of the models requiring specialists of differents fields --> synergy \newline
Insuffisant transparency \newline
Project openFRED (and short presentation of RLI?)\newline

\section{Hydropower}
Hydropower in the world \newline
Hydropower in Europe (example Norway?) \newline
Hydropower in Germany (places, part of the mix, installed capacity, potential) \newline
Pros and cons \newline

