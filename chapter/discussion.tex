\chapter{Discussion}
\label{chap:discussion}

In this work, an approach towards modeling run-of-the-river power plants and simulating electricity generation based on open access geodatabases was put forward. In this chapter, we will examine to what extent the model and the simulation process fit the specifications defined in section \ref{sub:spec}, discuss their potential applications, their limits, and give some ideas to improve the results.
\section{Consistency with the specifications}

The objectives of this work (see sec. \ref{sub:spec}) were to be able to obtain feed-in time series both from single plants and for large areas and to be compatible with the environment data sets of the OEDB (runoff, river network, ...). The developed model (\ref{chap:simulation_model}), together with the guidelines on preprocessing the data (sec. \ref{sec:data_preproc}), meet these requirements and allow the user to run simulations with various degrees of precision in the input data. However, it was difficult to assess the proper functioning of the model, due to the limited availability of production data. This is discussed further in section \ref{sec:limits}.

\section{Applications of the model}

Such a model finds applications at different levels. It can be a useful tool for an individual or a society seeking to build a run-of-the-river plant, as it reproduces the design process of a plant. The parameters related to human and economic decisions, such as the percentage of the year the plant should run at full load, can be changed to personalize the process. This is also true of the legal parameters such a the residual water flow. \newline
It can also be useful to research institutes or public bodies looking for a way to run state-wide simulations with little information about the plants installed. It could as well be integrated in energy systems modeling and optimization tools.

\section{Limits of the model}
\label{sec:limits}

If the model fulfills the conditions set in the specifications, the chapter presenting the results (chap. \ref{chap:results}) was not conclusive, and failed to lead to a validation or invalidation of the model. The single plant simulation seems to over-estimate the production while the state-wide simulation seems to under-estimate it. However in both cases the comparisons had important inconsistencies : the test for a single power plant was made over the running in period of the plant, during which it did not work to its full capacity, and the publicly available production values for federal states do not separate run-of-the-river hydroelectricity from other technologies (see sec. \ref{sec:db_hydroelec}). The model will have to be tested again when the HydroRaon power plant is on normal operating phase, and it would be interesting to have hydroelectricity production data separated technology by technology to run state-wide simulations with a more consistent basis. \newline
Moreover, section \ref{sec:missing_data} showed that values very far from the reality were assigned to the plants during the extrapolation process when done on the wrong runoff values (gauge station located on another river or far away from the plant, or raster cell not matching the river course). However, the assignment of an oversized nominal water flows leads to the assignment of a proportionally undersized water head. Thus, the power output stays consistent, as long as the runoff time series vary in the same order of magnitude as the river the plant should have been assigned to.

\section{Ways of improvement}
\label{sec:improv}

This work laid the foundations towards simulating run-of-the-river hydroelectricity within the openFRED project. Section \ref{sec:limits} and chapter \ref{chap:results} identified the limits of the developed model in its current state. However, these problems can be addressed.\newline
Section \ref{sec:limits} mentioned the problems appearing in the extrapolation process when a plant is assigned to runoff time series not consistent with the river it is actually in and explained why it should not impact the production too much, as long as the runoff time series vary in the same order of magnitude as the river the plant should have been assigned to. In the case of measured runoffs, this happens when plants are located on rivers with no gauge station. In this case, an approach more oriented towards hydrology could be used in the assignment of gauge stations, for instance by assigning a gauge from the same catchment area or river system, instead of looking for the nearest gauge. In the case of modeled water flows, the courses of the main rivers as seen by the raster could be identified, in order to assign the plants with a raster cell containing runoff time series consistent with the river. \newline
Section \ref{sub:extrapol_dV_n} showed how to extract the nominal water flow from the flow duration curve. The literature distinguishes isolated sites, for which the nominal water flow should be reached 250 days a year, and sites connected to the grid where the nominal water flow should be reached 50 to 90 days a year. In this work, the nominal water flows were chosen to be reached \unit[20]{\%} of the year (around 70 days) and this value seems coherent given the results from section \ref{sec:missing_data}. However, this parameter has a big influence on the overall production of a plant : a sensitivity analysis was made by simulating the production of Thuringia from modeled runoffs over a year (2008), with different values for this percentage :  with \unit[20]{\%} the yearly production is \unit[187]{GWh}, with \unit[30]{\%} it rises to \unit[210]{GWh} and with \unit[40]{\%} to \unit[223]{GWh}. This analysis shows that the percentage of the year were the plant works at full load has a big impact on the production and that this parameter cannot be set arbitrarily. It would be interesting to see if a correlation can be found between the nominal power and the nominal water flow : if \unit[20]{\%} seems to be an acceptable value for big plants like the ones on the Mosel, the smaller HydroRaon seems to have been designed to work at full load \unit[30]{\%} of the time.