\chapter{Methodology}
\label{chap:methodology}

\section{Literature review}
\subsection{Run-of-the-river hydropower}
VGB Powertech defines a run-of-the-river hydropower plant as a hydropower plant that uses the natural usable stream without delaying it \cite{vgb}. \newline
Quaschning \cite{quaschning} describes how a run-of-the-river power plant operates.Even though there is no reservoir accumulating the water upstream from the plant, run-of-the-river power plants need an altitude difference, called head, between the water surface before and after the plant. This can be achieved through a dam, a derivation of the water stream through a canal, or a lock \cite{tdi_petites_centrales}. The water is lead through a turbine (see figure \ref{schema_hpp}) that drives a electrical generator, generating a power proportional to the water flow and the head, as seen in equation \ref{eq_power_1} \cite{quaschning}.
\begin{equation}
\label{eq_power_1} 
 P = \rho_\mathrm{water} \cdot g \cdot Q_\mathrm{turbine} \cdot H \cdot \eta_\mathrm{turbine} \cdot \eta_\mathrm{generator}
\end{equation}
With 
\begin{itemize}
 \item $P$ \tabto{4cm} power produced by the generator \tabto{12cm} $[W]$
 \item $\rho_\mathrm{water}$ \tabto{4cm} density of water \tabto{12cm} $1000 \ kg/m^3$
 \item $g$ \tabto{4cm} acceleration of gravity \tabto{12cm} $9.81 \ m/s^2$
 \item $Q_\mathrm{turbine}$ \tabto{4cm} water flow through the turbine \tabto{12cm} $[m^3/s]$
 \item $H$ \tabto{4cm} head of water \tabto{12cm} $[m]$
 \item $\eta_\mathrm{turbine}$ \tabto{4cm} efficiency of the turbine
 \item $\eta_\mathrm{generator}$ \tabto{4cm} efficiency of the generator
\end{itemize}
\begin{figure}[h]
\includegraphics[width=16cm]{schema_hpp_de.png}
\caption[Schema of a run-of-the-river hydropower plant]{Schema of a run-of-the-river hydropower plant \cite{quaschning} XXXtranslate to englishXXX}
\centering
\label{schema_hpp}
\end{figure}

The head of water is the altitude difference between the surface of the river before and after the turbine. For run-of-the-river hydropower plants, it is assumed that the water level before the turbine is kept constant by the dam, while the water level downstream can vary. This happens when the waterflow exceeds the capacity of the turbine and has to be deviated over the dam. With this assumption, the head of water is given by the equation \ref{eq_head} and the waterflow in the turbine by the equation \ref{eq_waterflow} \cite{quaschning}.
\begin{equation}
\label{eq_head} 
 H = H_\mathrm{n} +W_\mathrm{n}-W
\end{equation}
\begin{equation}
\label{eq_waterflow} 
 Q_\mathrm{turbine} = min(Q_\mathrm{n},Q)
\end{equation}
Where $H_\mathrm{n}$, $W_\mathrm{n}$ and $Q_\mathrm{n}$ are respectively the nominal head, water level downstream from the dam and water flow, and W and Q the real water level and water flow.
\newline
The  equation \ref{eq_power_1} becomes :
\begin{equation}
 \label{eq_power_2} 
 P = \rho_\mathrm{water} \cdot g \cdot min(Q_\mathrm{n},Q) \cdot (H_\mathrm{n} +W_\mathrm{n}-W) \cdot \eta_\mathrm{turbine} \cdot \eta_\mathrm{generator}
\end{equation}

The efficiency factor of the generator depends on its power and on the part-load range \cite{pacer}. The swiss ``Bundesamt für Konjunkturfragen'' gives the values given in table \ref{eta_gen}.

\begin{table}
 \caption[Generator efficiency in full load and part load]{Generator efficiency in full load (left) and part load (right) \cite{pacer}}
 \label{eta_gen}
 \begin{tabular}{|C{3cm}|C{3cm}| C{2cm} |C{3cm}|C{3cm}|}
  \cline{1-2} \cline{4-5}
  $P_\mathrm{el} [kW]$ & $\eta_\mathrm{g,max}$  && $P_\mathrm{el}/P_\mathrm{el, max}$ & $\eta_\mathrm{g}/\eta_\mathrm{g,max}$ \\ 
  \cline{1-2} \cline{4-5}
  1 to 5 & 80\% to 85\% && 
  \multirow{4}{*}{\begin{tabular}{c}>50\%\\25\% \\10\% \\\end{tabular}}& 
  \multirow{4}{*}{\begin{tabular}{c}100\% \\95\% \\85\% \\\end{tabular}}\\
  5 to 20 & 85\% to 90\% &&& \\
  20 to 100 & 90\% to 95\% &&&\\ 
  More than 100 & 95\% &&&\\ 
  \cline{1-2} \cline{4-5}
\end{tabular}
\end{table}

The list of hydropower plants available at the Reiner Lemoine Institute only contains plants with a nominal power over 1 MW (see section \ref{hpp_register}). It is estimated that among the 6500 to 7500 hydropower plants in Germany, only 406 have a nominal power above 1MW \cite{uba_wasserkraft}. However, the plants under 1MW account for a small part of the total installed power (see figure \ref{uba_hpp}). For that reason, the generator efficiency has been approximated to 95\% for all power plants is this work.

\begin{figure}[H]
\includegraphics[width=16cm]{uba_hpp.png}
\caption[Installed power pro Bundesland for plants over and under 1MW]{Installed power pro Bundesland for plants over and under 1MW \cite{uba_wasserkraft} XXXtranslate to englishXXX}
\centering
\label{uba_hpp}
\end{figure}

The efficiency of the turbine depends on the type of turbine and on the part-load range \cite{quaschning}\cite{pacer}. There are four types of turbines used for run-of-the-river plants : Pelton, Francis, Kaplan and crossflow. These turbines have optimal performances over ranges of water flow and hydraulic head, and their application areas can be plotted on a characterisitc diagramm, as in figure \ref{charac_diag}. XXX Add other Quelle in AnhangXXX

\begin{figure}[H]
\includegraphics[width=16cm]{charac_diag.png}
\caption[Characteristic diagramm for several types of water turbines]{Characteristic diagramm for several types of water turbines \cite{wiki_WK} XXXtranslate to englishXXX}
\centering
\label{charac_diag}
\end{figure}

Figure \ref{efficiency_turb} give the efficiency curves of different turbine types depending on the part load. These efficiencies can be approximated by the empiric function given in equation \ref{eq_eff} with the parameters given in table \ref{eff_param} \cite{quaschning}. In this equation, $Q_\mathrm{min}$ is the minimal water flow to start the turbine, $Q_\mathrm{n}$ is the nominal waterflow of the turbine and $q=\frac{Q-Q_\mathrm{min}}{Q_\mathrm{n}}$.

\begin{equation}
 \label{eq_eff}
\eta_\mathrm{T}= \left\{
    \begin{array}{ll}
	0 & \mbox{for } Q \leq Q_\mathrm{min}\\
        \frac{q}{a_\mathrm{1}+a_\mathrm{2} \cdot q + a_\mathrm{3} \cdot q^2} & \mbox{for } Q_\mathrm{min}<Q<Q_\mathrm{n} \\
        \eta_\mathrm{T,n} & \mbox{for } Q \geq Q_\mathrm{n}
    \end{array}
\right.
\end{equation}


\begin{figure}[H]
\includegraphics[width=16cm]{efficiency_turb.png}
\caption[Efficiency of different types of turbines depending on the part load ratio]{Efficiency of different types of turbines depending on the part load ratio (water flow / nominal water flow) \cite{raa89} XXXtranslate to englishXXX}
\centering
\label{efficiency_turb}
\end{figure}


\begin{table}
 \caption[Parameters to calculate the turbine efficiency]{Parameters to calculate the turbine efficiency \cite{quaschning}}
 \label{eff_param}
 \centering
 \begin{tabular}{|c|c|c|c|c|c|}
  \cline{2-6}
  \multicolumn{1}{c|}{}&$Q_\mathrm{min} / Q\mathrm{n}$ & $\eta_\mathrm{T,n}$& $a_\mathrm{1}$ & $a_\mathrm{2}$&$a_\mathrm{3}$ \\ 
  \hline
  Kaplan & 0.081& 0.895& 0.045 &0.965& 0.1 \\
  Pelton & 0.07& 0.885& 0.03& 0.99& 0.1\\
  Francis &0.095 &0.89 &0.18 &0.63 &0.31 \\
  Propeller &0.42 &0.9 &0.25 &0.28 &0.69\\
  \hline
 \end{tabular}
\end{table}


\subsection{Hydropower plants registers}
\label{hpp_register}


OEDB, MaStr

\subsection{Modelling runoff}
Several works use equation \ref{eq_power_1} or \ref{eq_power_2} to calculate the production or potential of run-of-river hydroelectricity on a given river \cite{hammid} \cite{bayazit} \cite{garrido}. In the case of potential assesment, the head of water can be obtained through geographic information systems (GIS) by calculating the elevation difference between neighbouring cells. The runoff can also be assessed from weather data using GIS, as described by Döll in figure \ref{waterGAP}.
\begin{figure}[h]
\includegraphics[width=16cm]{waterGAP.png}
\caption[Representation of a global hydrological from the WaterGAP software]{Representation of a global hydrological fron the WaterGAP software \cite{doll}}
\centering
\label{waterGAP}
\end{figure}

\subsection{Measured runoff}
HYDABA
DAFG
Pegelonline
DJG
Banque hydro

\subsection{Other works}
Gesine usw

\section{Specifications of the model}

\subsection{Inputs}
Available data (HPP register oedb) and future available data (MaStr)
--> Pnenn, Standort, ...

\subsection{Outputs}
Zeitreihen pro Kraftwerk oder pro Ebene, mehrere Kraftwerke auf einmal simulieren


\subsection{Tools}
Python, GIS, databases